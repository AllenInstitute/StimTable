\subsection{Cleverref} \label{sec:intro_part}

This section demonstrates why cleverref is useful.

\begin{table}
    \caption{A table.}
    \begin{tabular}{ccc}
        Column & Column & Column \\ \hline
        One & Two & Three
    \end{tabular}
    \label{tab:exampletable}
\end{table}

\begin{align}
    E &= mc^2 \label{eqn:einstein} \\
    \frac{dz}{dt} &= \frac{z_\infty - z}{\tau} \label{eqn:nonsense}
\end{align}

Using cleverref we can refer to \cref{eqn:einstein}, \cref{sec:intro_part}, and
\cref{tab:exampletable} with a simple call to \verb|\cref{}| and let cleverref
figure out whether the environment being referenced is a section, table, etc.

\subsection{Acronyms}

\LaTeX glossaries automatically print the long-form version of an acronym the
first time it is used. Here's an example.

The \gls{drn} and \gls{mpfc} are brain regions with neurons. The \gls{drn}
is made up mainly of \gls{ser} cells. \Gls{som} neurons are found in both the
\gls{mpfc} and \gls{drn}.
